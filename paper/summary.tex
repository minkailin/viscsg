\section{Summary and discussion}\label{summary}
In this paper, we develop the linear theory of cooling, irradiated,
and turbulent accretion disks in order to understand gravitational
instability (GI) in realistic protoplanetary disks (PPDs). 
We model turbulence as an effective, 
Navier-Stokes viscosity. This viscosity provides a 
background heating to balance the imposed cooling,  
and may also be active in the perturbed state.  
We suggest that disk fragmentation, observed in numerical simulations,
can be understood as the non-linear outcome of a secondary
GI of the gravito-turbulent basic state, driven by cooling and/or viscosity.   
%This %allows us to assess realistic disk fragmentation without using
%uncertain empirical conditions based on numerical simulations.   
%{\bf cooling removes pressure support, viscosity removes rotational support}
%{\bf our conclusions do require input of emperical resuls from numerical
%sims}
%going beyond toomre 
%move to subsection?
While viscosity and cooling are related through thermal
balance, they independently enhance GI: cooling reduces thermal
stabilization; and viscous forces compromise rotational
stabilization. This provides a physical explanation as to why
rapidly-cooled, gravito-turbulent disks cannot exist. 

%these reduce Q 
%note: frag due to increase in sigma by turbulent fluctionations not modeled here 
%generalized dispersion relation (inviscid)
%no magic beta - connection with stochastic 
%cooling function includes irradiation 

The effect of cooling and irradiation on GI is quantified by the
dispersion relation 
Eq. \ref{inviscid}. The sufficient condition for stability 
depends on the irradiation level (Eq. \ref{stable_condition}), and 
\emph{independent} of the cooling rate, provided it is non-zero and
finite. This means that long cooling times can still  
formally lead to instability. However, growth timescales may be 
uninterestingly long for cooling times $\tcool\Omega\gtrsim
O(10)$. Because cooling affects pressure support, GI driven by cooling
occur on small scales, $kH\lesssim O(1)$.   

%We obtain an analytic expression for the
%dimensionless cooling time, $\beta_*$, required to remove pressure
%support against self-gravity on radial lengthscales comparable to the 
%disk scale height. We find $\beta_*$ reproduces the fragmentation
%boundary found in previous numerical simulations (Table
%\ref{bstar_compare}).  
%viscous GI (new: 3D, cooling, ext. heating)  

We generalize the `viscous gravitational instability', previously 
studied in isothermal disks
\citep{lynden-bell74,willerding92,gammie96}, to include 
cooling, viscous heating and irradiation. Our fiducial example,  
where the viscosity and self-gravity are inversely
related ($\alpha\propto Q^{-2}$), is a toy model for a
gravito-turbulent disk. We find viscous GI occurs on orbital  
timescales for $\alpha\gtrsim 0.1$. This is consistent with the notion
of a maximum stress sustainable by gravito-turbulence established in
previous numerical simulations. Because viscosity affects
rotational support, viscous GI occurs on large scales, $kH\gtrsim
O(1)$.  Furthermore, irradiation preferentially
stabilizes small-scale perturbations.     

%application to PPD %realistic relation between alpha and Q 

We apply our linear framework to protoplanetary disks with 
realistic models for cooling and gravito-turbulence
\citep{rafikov15}. We show that, for linear dynamics, realistic 
cooling in a physically non-irradiated disk ($T_\mathrm{irr}=0$) is
actually modeled by `beta' cooling with a  
non-zero irradiation parameter ($\theta\neq0$). However, 
numerical simulations typically employ beta cooling with 
$\theta=0$ \citep[e.g.][]{gammie01}. How relevant 
are such simulations to realistic PPDs must then be further 
investigated. The important implication of having 
$\theta\sim 1$ in a PPD is that cooling alone cannot 
destabilize the disk at large radii. Instead, viscous GI occurs on
dynamical timescales for $R\gtrsim 60$AU because 
$\alpha\gtrsim0.1$ there. This corresponds to a Toomre $Q\simeq 1.5$ and 
a cooling time $\tcool\lesssim 3\Omega^{-1}$. These are coincident with 
\emph{empirical} conditions cited in the literature to determine disk
fragmentation 
\citep[e.g.][]{rafikov15}. By contrast, we attribute a \emph{physical}
cause for the fragmentation of realistic PPDs: gravito-turbulent PPDs
fragment when turbulent stresses are large enough to further
destabilize the disk against self-gravity.    
%collary: details of cooling unimportant as long as the associated
%stress is the same 

%small scale transport 
%ultimately: why do disks fragment? viscosity may play a role
%details of thermo doesn't matter if it's viscosity that leads to
%fragmentation 

%we do not look at fragment survival
%fragment due to cooling or viscosity?
%they should produce diff size frags
\subsection{Relation to numerical simulations}\label{prev_works}
Our framework may shed light on 
%may offer an alternative understanding of 
some recent numerical results concerning disk
fragmentation.  \cite{paardekooper12} found that fragmentation can occur for 
slowly-cooled disks, $\tcool\Omega\gg O(1)$, but that this requires  
simulations to run for significantly longer than dynamical 
timescales. This is consistent with our finding
that for either cooling-driven or viscous GI, there is no critical 
cooling rate (viscosity) beyond (below) which the disk is  
formally stable. Instead, growth rates smoothly decrease with
increasing (decreasing) $\tcool$ ($\alpha$), implying that
instabilities take longer to develop for slowly-cooled disks.        
%hopkins paper 

%resolution convergence 
Another issue is convergence with numerical resolution. Notably, 
\cite{baehr15} found that for fixed cooling rates, disks eventually 
fragment with sufficient spatial resolution. %This could be due to 
%small-scale, turbulent angular momentum transport being enabled at
%high resolutions, which aids clump formation. 
This might be related to viscous GI in the following sense: high
resolutions enable small-scale turbulent angular momentum transport
away from over-dense regions, aiding clump formation. This 
mechanism is unavailable at low resolutions, for which one can only
resort to rapid cooling for destabilization.       
 
We note that power in gravito-turbulence is dominated by scales with
$kH\sim 1$ \citep{cossins09}, and this is always  
well-resolved by modern numerical simulations. However, this does not
necessarily imply small-scale dynamics are unimportant, as is evident
from the non-converging simulations described above.   

%it is true that most unstable wavelength is H is this is always
%resolved
%but that doesn't mean small-scale structures are unimportant 
%note: no turbulence observed inside clumps
%consistent with rho*nu = const  
%this alpha should be that associated with small-scale transport 
%frag due to cooling -> structures less than H
%frag due to viscosity -> structures more than H
%stochastic frag 
%non-convergence with resolution 
%no magic beta when non-ideal effects are included 

\subsection{MHD turbulence}%turbulence and viscosity 
It is important to note that while we consider gravito-turbulent disks  
as an application, the linear theory itself does not assume any 
specific origin of turbulence. 

%effect of MHD turbulence on GI 
Our models may thus apply to self-gravitating disks dominated by, for
example, magneto-hydrodynamic (MHD) turbulence.   
%In fact, the viscous approach may be more
%applicable here than gravito-turbulence, because the characteristic
%wavelength of the magneto-rotational instability (which leads to MHD
%turbulence) is $\ll H$, at least in the ideal MHD limit. Then there is
%a scale separation between the underlying, small-scale MHD 
%turbulence and the large-scale effects of self-gravity. 
%fromang is isothermal sims - no cooling effect 
Explicit numerical simulations of magnetized, massive disks have been performed 
by \cite{fromang05}. He finds disk fragmentation with increasing numerical
resolution, attributing this to resolving the most unstable MRI
wavelength, which enables small-scale angular momentum removal by MHD
turbulence to aid fragmentation. %this supports using viscosity to
                                %mimic turbulence
This would be a physical mechnism represented by the viscous GI
discussed in this paper; which lends some support for  
the use of a viscosity to represent turbulence.    

%these sims are isothermal so cooling is not responsible for frag 
%note fromang -> comparing HD and MHD at low res: 
%mhd turb lowers T_grav -> no frag
%suggest frag associated with stress 
%turb enhances frag 
%turb due to MHD 
%maybe even better model for (ideal) MHD turbulence 
%because mri scales are << H 
%scale-separation 
%application to gravito-turbulence but actually linear theory does not assume
%actually application to 
%MHD turbulence prob makes more sense 
%require `scale separation' for GI turbulence (small-scale GI provide
%turbulence, large-scale GI for fragmentation)

\subsection{Outstanding issues}
%the link between the linear GI's discussed in this paper 
%and frag 
True disk fragmentation is a non-linear process characterized by clumps   
reaching densities orders of magnitude above the ambient value, and
which survives tidal shear and shocks \citep{paardekooper12}.  
%cite the young papers 
Clearly, our linear models cannot address fragmentation
directly. Technically, we have only demonstrated that a 
gravito-turbulent state becomes dynamically unstable, and thus should
not persist, when cooling (viscosity) is too rapid (large).  
However, numerical simulations show that there are only two
possible outcomes of cooling, self-gravitating disks: steady
gravito-turbulence or fragmentation. It seems then reasonable to 
speculate that the non-existence of a gravito-turbulent state, here due to
dynamical instability, implies disk fragmentation.  

%modeling turb as visc %explore other visc laws 
The most important assumption in this work is modeling turbulence as a  
Navier-Stokes viscosity. Furthermore, we have chosen a particular  
viscosity law (see \S\ref{visc_model}) to mimic the effect of 
turbulence in reducing rotational support (in the sense that it
provides small-scale angular momentum transport). 
%to enhance clumping 
%this visc law is used in prev studies (by much smarter people)
How to quantitatively model turbulence (be it due to self-gravity or
MHD) as a viscosity, especially on dynamical timescales, should be
clarified with direct numerical simulations. The present linear models
should then be modified accordingly. 
%could consider other visc parameters mu and lambda (but this
%introduces visc overstability, or new functional forms of the
%viscosity law altogether 

%non-axisymmetry and transient dynamics 
Another possible generalization is non-axisymmetric
disturbances. In barotropic, inviscid disks  
non-axisymmetric GI can develop for $Q$ somewhat larger than unity
\citep{papaloizou89,papaloizou91}. It would be interesting to study
how non-axisymmetric perturbations are affected by cooling and
viscosity in order to improve %or verify 
the link between disk fragmentation and the
stability of gravito-turbulent disks.   

%papaloizou91, papaloizou lin 87 
%our linear model includes the same physics 
%we show 
%non-axisymmtric problem 
%viscosity model (maybe it doesn't behave like a power law)
%chosen a particular mu and lambda 
%link between linear instability and frag is issue
%true successful frag also depend on frag survival 

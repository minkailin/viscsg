\section{Summary and discussion}
In this paper, we study the implications on the stability of
turbulent, cooling accretion disks if said turbulence is modeled as an
effective, Navier-Stokes viscosity. This viscosity provides a 
background heating and may also be active in the perturbed 
state. We pay particular attention to 
protoplanetary disks subject to `gravito-turbulence'. 
Our framework provide an alternative interpretation of existing
results on PPD fragmentation established from direct numerical
simulations. 
{\bf cooling removes pressure support, viscosity removes rotational support}
{\bf our conclusions do require input of emperical resuls from numerical
sims}

%generalized dispersion relation (inviscid)
%no magic beta - connection with stochastic 

We derive the disperion relation for local axisymmetric density waves 
subject to external heating and cooling (Eq. \ref{inviscid}), but
without viscous effects on the waves. We find
the sufficient condition for stability 
is \emph{independent} of the cooling rate. Long cooling times can
formally lead to instability, but the growth timescales may be
uninterestingly long. We obtain an analytic expression for the
dimensionless cooling time, $\beta_*$, required to remove pressure
support against self-gravity on radial lengthscales comparable to the 
disk scale height. We find $\beta_*$ reproduces the fragmentation
boundary found in previous numerical simulations (Table
\ref{bstar_compare}).  

%viscous GI ad connection with frag

We then include viscosity in the linear problem and generalize the
`viscous gravitational instability', previously discussed in 
simpler disk models \citep{lynden-bell74,gammie96,hunter83}, to include
heating/cooling and three-dimensionality. Our generalized framework is
readily applied to realistic protoplanetary disk models. For
gravito-turbulent PPDs, we find viscous GI grows on orbital time-scales
beyond $\sim 100\mathrm{AU}$. This means the assumed gravito-turbulent
steady state should not exist beyond dynamical time-scales in the
outer parts of PPDs. 

%viscosity > 0.1, beta < 3 - 4 

Our results bear striking similarities with the current consensus on
PPD fragmentation. It is generally accepted that PPDs can fragment
beyond a few tens of $\mathrm{AU}$. This conclusion results from
applying a Toomre stability condition (e.g. $Q\lesssim 1.5$) and a
cooling time or viscosity criteria (e.g. $\beta\lesssim 3$ or
$\alpha\gtrsim 0.1$), the latter of which is
based on numerical simulations. We find these
emperical conditions essentially translates to viscous GI on dynamical
time-scales. This suggests that gravito-turbulent
PPDs fragment because the associated viscous stresses further
destabilize the disk against self-gravity. 


%ultimately: why do disks fragment? viscosity may play a role


% also conclude this region of 



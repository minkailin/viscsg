\section{Summary and discussion}\label{summary}
In this paper, we develop the linear theory of cooling, irradiated, 
and viscous accretion disks in order to understand gravitational 
instability (GI) in realistic protoplanetary disks (PPDs). 
We use a Navier-Stokes viscosity to mimic the effects of
turbulent angular momentum transport. 
%KMK
%which may be of any origin, though we specialize to
            %gravito-turb 
This viscosity provides a background heating to balance the imposed cooling,  
and may also act on linear perturbations. %thermo and dynamic 
We suggest that disk fragmentation observed in numerical simulations  
can be understood as the eventual outcome of secondary
instabilities of a gravito-turbulent base state, driven by cooling
and/or viscosity. 
%frag in sim and frag in real disk may or may not have the same
%physical origin


%KMK
Previous work has focused on the impact of viscosity and 
cooling on the equilibrium temperature and surface density of an 
accretion disk, but merely used this to calculate  the classic Toomre $Q$,
thereby assess stability. %Disk fragmentation is assessed similarly,
%but is usually based on empirically-derived conditions (see below). 
We demonstrate by explicitly including these effects into the
dispersion relation that they can drive secondary instabilities.
While viscosity and cooling can be related through thermal
balance, they independently enhance growth rates: cooling reduces
thermal stabilization; and viscous forces compromise rotational
stabilization. This provides a physical explanation as to why
rapidly-cooled, gravito-turbulent disks cannot exist. Moreover, we discuss below how
these models may lend support to the varied behavior observed in numerical simulations.
%KMK

%these reduce Q 
%note: frag due to increase in sigma by turbulent fluctionations not modeled here 
%generalized dispersion relation (inviscid)
%no magic beta - connection with stochastic 
%cooling function includes irradiation 

The effect of cooling and irradiation on GI is quantified by the 
dispersion relation Eq. \ref{inviscid}. 
We find sufficient conditions for stability which  
depends on the irradiation level (Eq. \ref{stable_condition}) but is 
\emph{independent} of the cooling time. 
This means that long cooling times can still  
formally lead to instability. However, growth timescales may be 
uninterestingly long for cooling times $\tcool\Omega\gtrsim 
O(10)$. Because cooling affects pressure support, GI driven by cooling
occur on small scales, $kH\gtrsim O(1)$.   

%We obtain an analytic expression for the
%dimensionless cooling time, $\beta_*$, required to remove pressure
%support against self-gravity on radial lengthscales comparable to the 
%disk scale height. We find $\beta_*$ reproduces the fragmentation
%boundary found in previous numerical simulations (Table
%\ref{bstar_compare}).  
%viscous GI (new: 3D, cooling, ext. heating)  

We generalize the `viscous gravitational instability', previously 
studied in isothermal disks
\citep{lynden-bell74,willerding92,gammie96}, to include 
cooling, viscous heating and irradiation. %and 3D 
We consider a disk with viscosity and self-gravity inversely
related ($\alpha\propto Q^{-2}$) to model a 
gravito-turbulent background, and find viscous GI occurs on orbital
timescales for $\alpha\gtrsim 0.1$. This is consistent with the notion 
of a maximum stress sustainable by gravito-turbulence established 
by numerical simulations \citep{rice05}. Because viscosity affects 
rotational support, viscous GI occurs on large scales, $kH\lesssim
O(1)$.  Furthermore, irradiation preferentially
stabilizes small-scale perturbations. %also true for inviscid case
%maybe that's why irradiated disks produce smaller alpha     
%%KMK this might be worth saying
                             
%application to PPD %realistic relation between alpha and Q 

We apply our linear framework to protoplanetary disks with 
realistic models for cooling and gravito-turbulence. 
We show that with a physically motivated cooling model for PPDs, 
cooling alone does not lead to gravitational instabilities. This is
due to stabilization by an effective `irradiation' associated with the 
density-dependence of the PPD cooling function as it appears in the
stability problem, which is present even if there is no physical
irradiation. This captures the fact that density enhancements impede cooling. 
 
%This effective irradiation arises because, for the stability problem,
%cf eqm disk 
%a non-zero floor temperature is generally required in the cooling
%prescription (Eq. \ref{beta_cool}) in order to map it to PPD cooling,
%regardless of the actual irradiation temperature.     

%it's associated with density-dependence 

%is because PPD cooling depends on both pressure and
%temperature. Mapping this 
%to the standard beta cooling function requires a non-zero floor
%temperature, otherwise standard beta c    

%This can only be mapped to the
%standard beta cooling 
%Physically, this  term
%represents the fact that disks do not cool towards zero temperature because
%radiative cooling is a strong (quartic) function of temperature  %%KMK?
%due to matching functional forms of
                                %cooling 

Instead, viscous GI occur on dynamical timescales in a PPD for
$R\gtrsim 60$AU because $\alpha\gtrsim0.1$ there. 
This corresponds to a Toomre $Q\simeq 1.5$ and 
a cooling time $\tcool\lesssim 3\Omega^{-1}$. These are coincident with 
\emph{empirical} conditions cited in the literature to determine disk
fragmentation 
\citep[e.g.][]{rafikov15}. Here, we attribute a \emph{physical}
cause for the fragmentation of realistic PPDs: gravito-turbulent PPDs
fragment when turbulent stresses are large enough to further
destabilize the disk against self-gravity.    
%collary: details of cooling unimportant as long as the associated
%stress is the same 

%small scale transport 
%ultimately: why do disks fragment? viscosity may play a role
%details of thermo doesn't matter if it's viscosity that leads to
%fragmentation 

\subsection{Relation to numerical simulations}\label{prev_works}
%frag in sims because of : cooling, viscosity, or high den
%flucuations? don't know 
Our results may help understand some numerical simulations 
concerning disk fragmentation. Table 
\ref{bstar_compare} shows a close match between the characteristic
cooling time for cooling-driven GI (Eq. \ref{betastar}) 
and that for disk fragmentation observed in simulations 
\citep{gammie01,rice05,rice11}. This suggests that, at least for those
simulations, fragmentation is physically due to the removal of
thermal stabilization by cooling on radial lengthscales of the disk 
thickness. In this interpretation, gravito-turbulence only
provides a background heating. %not active in frag 
Our characteristic cooling time corresponds to a dimensionless background viscosity 
as defined in the above studies\footnote{This differs from our
  definition of $\alpha$ by a factor of $\gamma^{-1}$.}  
\begin{align}
  \alpha =
  \frac{4}{9}\frac{\left(\sqrt{\gamma}-1\right)^{1/2}}{\gamma\left(\sqrt{\gamma}+1\right)}
  \simeq \begin{cases}
    0.062 & \gamma = 7/5, \\
    0.063 & \gamma = 5/3,\\
    0.059 & \gamma = 2.
  \end{cases}
\end{align}
This $\alpha\sim0.06$ is roughly constant, consistent with \cite{rice05}. 

%where we have inserted a factor of $\gamma^{-1}$ to fit the
%viscosity definition used by \cite{gammie01}. 

%i.e. there is no effective heating mechanism on the scale of
        %H 
%{\bf This would imply over-densities on this scale 
%do not experience heating during its growth. This is not unreasonable
%because in reality, gravito-turbulent heating only occurs when a 
%spiral shock passes through said over-density \citep{young16}. Thus
%there there is a time interval during which only cooling acts, whence
%our inviscid theory can be applied. --- too much?}

%%KMK
%Note that unlike our physical cooling function (Eq. \ref{real_cool}),
%the standard beta function in these 
%numerical simulations cools towards zero temperature, rather than a
%temperature floor or irradiation temperature  
%%KMK some comment like that here, or too redundant with comment below??
 %i.e. cool-driven g  

%{\bf don't know if we want to dwell on this table}
%negligible effect from viscous heating
%makes sense? sims show clumps are not turbulent internally
%then visc plays no role in clumping -> only a cooling effect 

More recent simulations have raised the issue of numerical
convergence. %KMK 
\cite{meru11} found that better resolved disks fragmented at longer cooling times.
Follow-up studies attributed at least some of this effect to decreasing numerical viscous
heating at higher resolution, which helps fragmentation
\citep{lodato11,meru12}. 
We can expect this if numerical viscosity
contributes to an effective irradiation, %then we do expect a disk can be
%destabilized at longer cooling times with increasing resolution, 
because then perturbations can cool to lower temperatures with
increasing resolution, see Fig. \ref{invisc_theta}.  {\bf
  In global simulations, non-convergence has also been attributed to initial 
  conditions that lead to internal edges \citep{paardekooper11b}, but
  this cannot be modeled in our local setup. 
  \cite{rice12} point out that the standard implementation of beta cooling in
  SPH simulations apply cooling on scales well-below the SPH smoothing
  lengths, and that this inconsistency may contribute to
  non-convergence. 
}

{\bf However,} \cite{paardekooper12} also found {\bf in local 2D
  grid-based simulations} that fragmentation can occur  
for  slowly-cooled disks with $\tcool\Omega\gg O(1)$, but that this requires   
simulations to run for significantly longer than dynamical 
timescales. This is consistent with our finding
that for either cooling-driven or viscous GI, there is no critical 
cooling rate/viscosity below  which the disk is  
formally stable. Instead, growth rates smoothly decrease with
increasing $\tcool$ (decreasing $\alpha$), implying that
instabilities, and hence fragmentation, simply take longer to develop
for slowly-cooled disks.         
%hopkins paper 

%%KMK moved up 
%resolution convergence 
%Another issue is convergence with numerical resolution. For example,
%cite{meru11} found that the critical cooling time for 
%disk fragmentation increases with numerical resolution. %sph particles in
% their case 
%Follow-up studies attributed this to decreasing numerical viscous
%heating at higher resolution, which helps fragmentation
%\citep{lodato11,meru12}. %also global effects in
                         %these. (paardekooper/meru/baruteau) 
%note that these sims do not have a floor temp
%is there an effective floor temp related to num visc?
%high res -> smaller effective floor temp -> easier to frag? 

Here, we highlight that most numerical experiments, including those
above, employ the standard beta cooling function, Eq. \ref{beta_cool},
\emph{without} a physical floor temperature. We show in \S\ref{2d_inviscid}
(see also \S\ref{cool_gi}) that this implies cooling-driven GI can
occur at any sufficiently small scale. Therefore as the numerical
resolution increases, simulations can access a wider range of unstable
scales. Although small-scale modes have weaker growth rates, they can 
become important over long timescales.   
%can access smaller, unstable scales 
In this respect, it is perhaps not surprising to find non-convergence with 
increasing resolution and/or integration times. 
{\bf
  On the other hand, the convergence issue may be
  less serious in 3D since small-scale modes are more stable in 3D
  than in 2D (Appendix \ref{3dcorr}). 
}
%simple beta cooling will not give a
%well-imposed problem 

We suggest having a physical floor temperature in the standard beta
cooling prescription is  necessary for  
numerical convergence. This limits the
relevant scales to a finite range.  %for cool gi 
%physically irradiation 
Furthermore, without a floor temperature, standard beta cooling is
a function of the pressure/energy density field only. 
%It is not clear 
%how relevant are results obtained from this is to actual PPDs,
There may be some inconsistency in applying results obtained from 
this to actual PPDs where cooling depends on two thermodynamic states (e.g. pressure and
temperature). A floor temperature permits a mapping between standard beta
cooling and PPD cooling (\S\ref{ppd_cooling}). 
%KMK
Moreover, the standard beta cooling does 
not account for optical depth effects, which force the mid-plane 
and high density perturbations to cool more slowly.

%ultra-small scale cooling-GI may have slow growth rates
%but will eventually matter if you sim for a long time 
%
%limit the relevant scales to a finite range 
%we suggest that having a floor temperature (irradiation) is necessary
%(though may not sufficient) 
%floor temp may be physical: irradiation, or numerical (sph particles
%can't have zero jitter)

%point out viscosity can have a dynamic role in helping frag 

%KMK
A temperature floor may be a necessary, but not sufficient condition
for numerical convergence. \cite{baehr15} included a floor temperature 
in their {\bf local 2D} simulations but still find that at fixed cooling rates, disks
eventually fragment with sufficient spatial resolution.
In light of our results on viscous GI, 
we suggest another possible contribution to non-convergence: %in addition to num visc (meru lodato)
high resolution enables small-scale turbulent angular momentum transport 
% physical, cf numerical  
to aid clump formation via the removal of rotational support. (See also \S\ref{MHD} below.) 
If clumps are only marginally resolved, simulations may not have sufficient dynamic range for turbulent eddies
to cascade down to these scales.
%This mechanism is unavailable at low resolutions, for which one must
%rely on rapid cooling.  
%maybe there is an effective floor temp which depends on res 

We comment that although modern simulations resolve the dominant
scale associated with gravito-turbulence very well, $kH\sim 1$
\citep{cossins09}, this does not
necessarily imply small-scale dynamics/thermodynamics are unimportant 
\citep[especially for non-linear evolution, ][]{young15},  
as is evident from the non-converging simulations described above.   
%KMK
Even at very high resolution one might worry that the artificial dissipation scale
imposed by the grid/smoothing length is too similar to the scale of fragmentation.
%it is true that most unstable wavelength is H is this is always
%resolved
%but that doesn't mean small-scale structures are unimportant 
%note: no turbulence observed inside clumps
%consistent with rho*nu = const  
%this alpha should be that associated with small-scale transport 
%frag due to cooling -> structures less than H
%frag due to viscosity -> structures more than H
%stochastic frag 
%non-convergence with resolution 
%no magic beta when non-ideal effects are included 

\subsection{MHD turbulence}\label{MHD}%turbulence and viscosity 
We emphasize that the linear framework we have developed does not assume  
a particular origin for the turbulence that is represented by the imposed
viscosity. For example, our 2D dispersion 
relation, Eq. \ref{thindisk} with Eq. \ref{bigA}---\ref{bigF}, treats 
$\alpha$ as an independent input parameter. 

%We only considered gravito-turbulence as an application.  
%In
%practice, this amounts to relating the $\alpha$ parameter to $Q$. 
%most equations treat alpha/Q/beta as independent

%effect of MHD turbulence on GI 
Our models may thus apply to self-gravitating disks dominated by MHD
turbulence.     
%In fact, the viscous approach may be more
%applicable here than gravito-turbulence, because the characteristic
%wavelength of the magneto-rotational instability (which leads to MHD
%turbulence) is $\ll H$, at least in the ideal MHD limit. Then there is
%a scale separation between the underlying, small-scale MHD 
%turbulence and the large-scale effects of self-gravity. 
%fromang is isothermal sims - no cooling effect 
Explicit numerical simulations of magnetized, massive disks have been performed 
by \cite{fromang05}. This study finds disk fragmentation with increasing numerical
resolution, and attributes this to resolving the most unstable MRI
wavelength, which enables small-scale angular momentum removal by MHD
turbulence to aid fragmentation. %clumping/mass accretion onto clump
This physical mechanism is represented by the viscous GI 
discussed in this paper, which lends some support for  
the use of a viscosity to represent turbulence.    

%these sims are isothermal so cooling is not responsible for frag 
%note fromang -> comparing HD and MHD at low res: 
%mhd turb lowers T_grav -> no frag
%suggest frag associated with stress 
%turb enhances frag 
%turb due to MHD 
%maybe even better model for (ideal) MHD turbulence 
%because mri scales are << H 
%scale-separation 
%application to gravito-turbulence but actually linear theory does not assume
%actually application to 
%MHD turbulence prob makes more sense 
%require `scale separation' for GI turbulence (small-scale GI provide
%turbulence, large-scale GI for fragmentation)

\subsection{Outstanding issues}
%the link between the linear GI's discussed in this paper 
%and frag 
True disk fragmentation is a non-linear process characterized by
clumps reaching densities that are orders of magnitude above the
ambient value. They must also survive 
disruption by tidal shear and shocks \citep{shlosman87,young16}.  
Clearly, our linear models cannot address fragmentation
directly. Technically, we have only demonstrated that a 
gravito-turbulent state becomes dynamically unstable, and thus should
not persist, when cooling is too rapid or when the associated viscous
stresses are too large.  However, steady gravito-turbulence or 
fragmentation are the \emph{only} possible outcomes of cooling, 
self-gravitating disks in the local limit \citep[][]{gammie01}.    
Thus it seems reasonable to speculate that the non-existence of a stable
gravito-turbulent state, here due to dynamical instability, implies
disk fragmentation.

%Our goal is to highlight how non-ideal effects, such as viscosity and
%cooling, can promote GI. However, the Toomre parameter
%(Eq. \ref{toomreQ_criterion}) suggest a third possibility: increasing
%the surface density. 

Our deterministic approach cannot model `stochastic fragmentation'
\citep{paardekooper12,hopkins13}. In this interpretation,  
fragmentation is attributed to the occurrence and survival 
of large, non-linear density enhancements, which arise from the
gravito-turbulent fluctuations simply by chance. %KMK
There is insufficient evidence that gravito-turbulence adequately samples
the density power spectrum as assumed by \cite{hopkins13}.
%cooling affects survival rate 
Nevertheless, one might consider this form of fragmentation as a secondary instability
triggered by lowering the local 
Toomre parameter through a (random) increase in density. 
%(Eq. \ref{toomreQ_criterion}).
% By contrast, the viscosity/cooling
%destabilizing mechanisms discussed in this paper lowers $Q$ by
%reducing pressure and Coriolis forces. 
%{\bf KMK: this last sentence sounds like it's just about changing classic Q}.
%we have a physical way of lowering pressure/rotation 

%stochasitc frag -> it just happens by chance. no physics 

%modeling turb as visc %explore other visc laws 
The most important assumption in this work is modeling turbulence as a  
Navier-Stokes viscosity. Furthermore, we have chosen a particular  
viscosity law (see \S\ref{visc_model}) to mimic the effect of 
turbulence in reducing rotational support (in the sense that it
provides small-scale angular momentum transport). 
%to enhance clumping 
%this visc law is used in prev studies (by much smarter people)
How to quantitatively model the effect of gravito- or MHD turbulence  
as a viscosity, especially on dynamical timescales, should be 
clarified with direct numerical simulations. The present viscosity
models should then be modified accordingly. 
%could consider other visc parameters mu and lambda (but this
%introduces visc overstability, or new functional forms of the
%viscosity law altogether 

%non-axisymmetry and transient dynamics 
%While our linear problem has been formulated with some 
Another possible generalization is non-axisymmetric
disturbances. In barotropic, inviscid disks  
non-axisymmetric global GI can develop for $Q$ somewhat larger than unity
\citep{papaloizou89,adams89,papaloizou91,laughlin97}. It would be interesting to study
how non-axisymmetric perturbations are affected by cooling and
viscosity in order to improve %or verify 
the link between disk fragmentation and the
stability of gravito-turbulent disks.   

%papaloizou91, papaloizou lin 87 
%our linear model includes the same physics 
%we show 
%non-axisymmtric problem 
%viscosity model (maybe it doesn't behave like a power law)
%chosen a particular mu and lambda 
%link between linear instability and frag is issue
%true successful frag also depend on frag survival 


%large stresses -> viscous GI operates quickly -> get non-linear den perts quickly 

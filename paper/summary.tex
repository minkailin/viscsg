\section{Summary and discussion}\label{summary}
In this paper, we develop the linear theory of cooling,
turbulent accretion disks in order to understand gravitational
instability (GI) in realistic protoplanetary disks (PPDs). 
We model turbulence as an effective, 
Navier-Stokes viscosity. This viscosity provides a 
background heating to balance the imposed cooling rate,  
and may also be active in the perturbed state. We provide physical
interpretations of disk fragmentation, observed in 
numerical simulations, as a secondary GI from a 
gravito-turbulent state, driven by cooling and/or viscosity. 
%{\bf cooling removes pressure support, viscosity removes rotational support}
%{\bf our conclusions do require input of emperical resuls from numerical
%sims}
%going beyond toomre 
%move to subsection?
While viscosity and cooling are inversely related through thermal
balance, they independently enhance GI: cooling reduces thermal
support; and viscous forces compromise rotational support. 

%generalized dispersion relation (inviscid)
%no magic beta - connection with stochastic 

The effect of cooling on GI is quantified by the dispersion
relation Eq. \ref{inviscid}. 
%We derive the disperion relation for local axisymmetric density waves 
%subject to external heating and cooling (Eq. \ref{inviscid}), but
%without viscous effects. 
We find
the sufficient condition for stability is \emph{independent} of the
cooling rate, provided it is non-zero and finite.   
Long cooling times can
formally lead to instability, but the growth timescales may be
uninterestingly long. 

%We obtain an analytic expression for the
%dimensionless cooling time, $\beta_*$, required to remove pressure
%support against self-gravity on radial lengthscales comparable to the 
%disk scale height. We find $\beta_*$ reproduces the fragmentation
%boundary found in previous numerical simulations (Table
%\ref{bstar_compare}).  


%viscous GI (new: 3D, cooling, ext. heating)  

We generalize the `viscous gravitational instability', previously discussed in 
simpler disk models \citep{lynden-bell74,gammie96,hunter83}, to include
heating/cooling and three-dimensionality. 



%application to PPD

Our generalized framework is
readily applied to realistic protoplanetary disk models. For
gravito-turbulent PPDs, we find viscous GI grows on orbital time-scales
beyond $\sim 100\mathrm{AU}$. This means the assumed gravito-turbulent
steady state should not exist beyond dynamical time-scales in the
outer parts of PPDs. 

Our results bear striking similarities with the current
consensus on PPD fragmentation. It is generally ac-
cepted that PPDs can fragment beyond a few tens of
AU. This conclusion results from applying a Toomre
stability condition (e.g. $Q\lesssim 1.5$) and a cooling time or
viscosity criteria (e.g. $\beta\lesssim3$ or $\alpha\gtrsim 0.1$), the latter of
which is based on numerical simulations. We find these
emperical conditions essentially translates to viscous GI
on dynamical time-scales. This suggests that gravito-
turbulent PPDs fragment because the associated viscous
stresses further destabilize the disk against self-gravity.


%ultimately: why do disks fragment? viscosity may play a role
%details of thermo doesn't matter if it's viscosity that leads to
%fragmentation 


\subsection{Relation to numerical simulations}\label{prev_works}







%frag due to cooling -> structures less than H
%frag due to viscosity -> structures more than H
%stochastic frag
%non-convergence with resolution 

\subsection{MHD turbulence}
%application to gravito-turbulence but actually linear theory does not assume
%actually application to 
%MHD turbulence prob makes more sense 
%require `scale separation' for GI turbulence (small-scale GI provide
%turbulence, large-scale GI for fragmentation)

\subsection{Outstanding issues}
%non-axisymmtric problem 
%viscosity model (maybe it doesn't behave like a power law)
%chosen a particular mu and lambda 

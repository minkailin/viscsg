\section{2D dispersion relation}\label{2ddisp}
The functions in Eq. \ref{thindisk} are 
\begin{align}
  A(S,K) =& \left(\frac{4}{3}\alpha+\alpha_b\right) K^2 + S +
  \mathcal{E}\mathcal{F}\label{bigA}
  - \frac{2|K|}{QS}, \\
  B(S,K) =& 2\left(\alpha q \mathcal{F} - 1\right),\\
  C(S,K) =& (2 - q) + \frac{\alpha q K^2(1+\mu)}{S} 
  + \alpha q \lambda \mathcal{E}\mathcal{F},\\
  D(S,K) = & \alpha K^2 + S + 2\alpha^2q^2\lambda\mathcal{F},
\end{align}
with
\begin{align}
  \mathcal{E} = \frac{\alpha q^2(1+\mu)}{S} +
    \frac{\gamma}{\gamma-1} + \frac{\theta}{\beta S(\gamma-1)},\quad
  \mathcal{F} = \frac{K^2(\gamma-1)\beta}{1 + \beta S - \alpha\beta
    q^2\lambda(\gamma-1)}\label{bigF}. 
\end{align}

\section{2D viscous GI} \label{gammie_check}
We obtain the dispersion relation for viscous GI described by
previous authors \citep{lynden-bell74,willerding92,gammie96} as 
follows. We set $\mu=-1, \lambda=0$ to obtain the same viscosity
models. Next we consider $|\beta S|\to \infty$, i.e. no cooling. Then
the condition $AD = BC$ imply
\begin{align}
  S^3 + \left(\frac{7}{3}\alpha + \alpha_b\right)K^2S + \left[2(2-q) -
    \frac{2|K|}{Q} + \gamma K^2 + \alpha K^4 \left(\frac{4}{3}\alpha +
    \alpha_b\right)\right] + \alpha K^2 \left[\gamma K^2 -
    \frac{2|K|}{Q} - 2q(2-q)(\gamma-1)\right]=0,
\end{align}
which agrees with \citet[Eq. 11]{willerding92} in the isothermal
limit ($\gamma=1$). The non-isothermal term $\propto (\gamma-1)$
originates from viscous dissipation which was excluded in the
aforementioned works. Its effect is to increase 
(decrease) the maximum wavenumber (wavelength) allowed for viscous
GI. 

An 

%We recover \cite{gammie96}'s dispersion relation 
%for viscous gravitational instability as follows.  Next we
%simplify Eq. \ref{bigA}---\ref{bigF} by assuming $|\beta S |\gg 1$  
%This will turn out to be equivalent to $\beta\gg 1$, i.e. the
%adiabatic limit. 
%and $S\lesssim O(K^2)$ as $|K|\to 0$. Applying these considerations to
%Eq. \ref{thindisk} gives 
%\begin{align}
%  S = \frac{\alpha K^2\left[\frac{2|K|}{Q} - \gamma K^2 +
%      2q(2-q)(\gamma-1)\right]}{2(2-q) + \gamma K^2 - \frac{2K}{Q}}, 
%\end{align}
%which does not explicitly depend on cooling. 
%Setting $\gamma=1$ then gives \citeauthor{gammie96}'s Eq. 18. The above
%is expression thus generalizes \citeauthor{gammie96}'s result to
%non-isothermal disks. 

%cooling is not the physical cause of instability 

\section{3D corrections in 2D theory}\label{3dcorr}
The simplest way to mimic the effect of finite disk thickness
on gravitational instabilities is to weaken self-gravity by reducing 
the gravitational constant  
\begin{align}
  G \to G\left(1+|k|H_\mathrm{sg}\right)^{-1}, 
\end{align}
or equivalently $Q\to Q\left(1+|k|H_\mathrm{sg}\right)$. This
prescription, derived by \cite{shu84}, is widely applied
\citep[e.g.][]{youdin11,takahashi14}. Here $H_\mathrm{sg}$ is a
measure of the disk thickness. We intuitively expect
$H_\mathrm{sg}\sim H$, but its precise value is not known a priori.   
We regard $H_\mathrm{sg}$ as a free parameter of the problem.    



\section{Basic equations}
We consider a three-dimensional (3D), self-gravitating,
viscous disk with heating and cooling. We use the shearing box framework to
study a small patch of the disk \citep{goldreich65}. The local frame
co-rotates with a fiducial point in the unperturbed disk at angular
frequency $\Omega$. The Cartesian co-ordinates $(x,y,z)$ corresponds
to the radial, azimuthal and verical directions in the global disk. 
The fluid equations are
\begin{align}
  &\frac{\p\rho}{\p t} + \nabla\cdot\left(\rho\bm{v}\right) = 0, \\
  & \frac{\p\bm{v}}{\p t} + \bm{v}\cdot\nabla\bm{v} =
  -\frac{1}{\rho}\nabla P - \nabla\Phi - 2\Omega\hat{\bm{z}}\times\bm{v}\notag\\ & 
  \phantom{\frac{\p\bm{v}}{\p t} + \bm{v}\cdot\nabla\bm{v} =}
  +2\Omega^2qx\hat{\bm{x}} -\Omega_z^2z\hat{\bm{z}} 
   + \frac{1}{\rho}\nabla\cdot\bm{T},\label{momentum_eq}\\
  & \frac{\p E}{\p t} + \nabla\cdot\left(E\bm{v}\right) = - P
  \nabla\cdot\bm{v} + \mathcal{H}_\mathrm{visc} +
  \mathcal{H}_\mathrm{ext} - \Lambda,\label{energy_eq}
\end{align} 
where $\rho$ is the density field and $\bm{v} = (v_x, v_y, v_z)$ is
the velocity field.% $q$ is the shear parameter and $\Omega_z$ is the
%local vertical oscillation frequency. 
We assume an ideal has so that the pressure $P$ and thermal energy
density $E$ are related by  
\begin{align}
  P = (\gamma-1)E = \mathcal{R}\rho T, 
\end{align}
where $\gamma$ is the adiabatic index, $\mathcal{R}$ is the gas
constant, and $T$ is the temperature. We use $\gamma=1.4$ unless
otherwise stated. The gas 
gravitatational potential $\Phi$ is given via the Poisson equation,  
\begin{align}
 \nabla^2\Phi = 4 \pi G \rho, 
\end{align}
where $G$ is the gravitational constant.  

In the momentum equation (Eq. \ref{momentum_eq}), the third,
fourth/fifth, and last term on the right-hand side represent the Coriolis, 
tidal, and viscous forces (see below), respectively. We consider    
Keplerian disks with shear paramter $q=3/2$ and vertical oscillation 
frequency $\Omega_z=\Omega$.  

In the energy equation (Eq. \ref{energy_eq}) the source terms 
$\mathcal{H}_\mathrm{visc}$, $\mathcal{H}_\mathrm{ext}$ and $\Lambda$
represent viscous heating, time-independent external heating and
thermal losses, respectively, described below. We normally set
$\mathcal{H}_\mathrm{ext}=0$ unless otherwise stated. 

\subsection{Viscosity and heating}\label{visc_model}
The Cartesian components of the viscous stress tensor $\bm{T}$ is
defined by 
\begin{align}
  T_{ij} \equiv \rho \left[\nu \left(\p_j v_i + \p_i v_j\right) +
    \left(\nu_b-\frac{2}{3}\nu\right)\delta_{ij}\nabla\cdot\bm{v}\right], 
\end{align}
%bulk visc ignored 
where $\rho\nu$ is the shear viscosity, and we include a bulk
viscosity $\rho\nu_b$ for completeness.  
The associated viscous heating given by
\begin{align}
  \mathcal{H}_\mathrm{visc} \equiv \left(\p_jv_i\right)T_{ij}, 
\end{align}
where summation over repeated indices is implied. 

We adopt the viscosity law  
\begin{align}
  \nu = \alpha
  \frac{c_{s0}^2}{\Omega}\left(\frac{\rho}{\rho_\mathrm{eq}}\right)^\mu\left(\frac{P}{P_\mathrm{eq}}\right)^\lambda,            
\end{align}
where subscript `eq' denotes the equilibrium state and  
%$c_{s0}$ is a modified sound-speed defined below. 
$c_{s0}^2\equiv P_\mathrm{eq}(0)/\rho_\mathrm{eq}(0)$. 
The dimensionless coefficient
$\alpha=\alpha(\rho_\mathrm{eq},P_\mathrm{eq})$ characterizes the
magnitude of the shear viscosity and is time-independent.  
The indices $\mu,\,\lambda$ are chosen to model how 
the viscosity behaves in the perturbed state. Given the uncertain
properties of gravito-turbulence, we will treat $\mu,\lambda$ as free
parameters. We adopt the same prescription for the bulk viscosity but
with $\alpha\to\alpha_b$.   

Our numerical calculations mostly use $\mu=-1,\lambda=0$ so
that $\rho\nu$ remains constant, as considered
in previous gravitational instabilities with 
viscosity \citep{hunter83,gammie96}. 
This model conveniently eliminates  
viscous over-stability \citep{schmit95,latter06}, which is unrelated
to self-gravity, that would otherwise contaminate our results. 

While the choice $\mu=-1,\lambda=0$ may seem artifical, we emphasize
this is done in order to bring out the relevant physical process ---
viscosity-assisted gravitational instability. Interestingly, 
\cite{laughlin96b} had already suggested a $1/\rho$ dependence for the 
shear viscosity when modeling the evoluting of self-gravitating disks
as a viscous process.  
{\bf note paardekooper 2012: alpha goes down as Sigma goes up}

\subsection{Steady states and cooling models}

We consider equilibrium solutions (omitting the `eq' subscripts
hereafter for clarity)
\begin{align}
  \bm{v} &= -q\Omega x \hat{\bm{y}}, \\
  \rho   &= \rho(z),\\
  P      &= P(z) \equiv c_s^2(z)\rho.
\end{align} 
%and assume a polytropic relation for the equilibrium pressure,
%\begin{align}
%  P  =
%  \frac{c_{s0}^2\rho_0}{\Gamma}\left(\frac{\rho}{\rho_0}\right)^\Gamma
%  \equiv \frac{c_s^2(z)}{\Gamma}\rho,
%\end{align}
%where $\rho_0$ is the equilibrium midplane density and $\Gamma$ is the
%constant polytropic index.  
The equilibrium density and pressure fields are obtained by solving
the vertical momentum equation with self-gravity,
\begin{align}
  &\frac{1}{\rho}\frac{dP}{dz} +
  \Omega_z^2z + \frac{d\Phi}{dz} = 0, \label{vert_eq1}\\
 &\frac{d^2\Phi}{dz^2} = 4 \pi G \rho,\label{vert_eq2}
\end{align}
%A numerical solution is usually required to obtain the equilibrium
%state, except for special values of $\Gamma$. 
together with thermodynamic equilibrium,
%\subsection{Thermal equilibrium}
%In equilibrium the energy equation implies
\begin{align}\label{thermal_eq}
%  \mathcal{H}_\mathrm{visc} + \mathcal{H}_\mathrm{ext} = \Lambda.   
(q\Omega)^2\rho\nu + \mathcal{H}_\mathrm{ext} = \Lambda,
\end{align}
where the first term represents viscous heating. For the viscous
problem we may set $\mathcal{H}_\mathrm{ext}=0$. However, if we wish
to neglect viscosity but include cooling, we must invoke
$\mathcal{H}_\mathrm{ext}\neq0$ to define an equilibrium state.  
To proceed further, we separately describe the two cooling models
considered in this work. 
%It is possible to have both models present - but too complicated 

\subsubsection{Beta cooling}
In this case the energy loss per unit volume is specified 
as an explicit function of the thermodynamic variables. 
A prototypical example is 
\begin{align}\label{beta_cool}
  \Lambda =
  %  \frac{P}{(\gamma-1)T}\frac{\left(T-T_\mathrm{irr}\right)}{t_c}, 
  \frac{\mathcal{R}\rho}{(\gamma-1)}\frac{\left(T-T_\mathrm{irr}\right)}{t_c}, 
\end{align}
where 
\begin{align}
  \tcool = \beta\Omega^{-1} 
\end{align}
is the cooling timescale with $\beta$ a constant input parameter, and
\begin{align}
  \tirr = \theta T_\mathrm{eq}
\end{align}
is a reference temperature field parameterized as a constant fraction
$\theta$ of the equilibrium temperature $T_\mathrm{eq}$. We will refer
to $\theta$ as the `irradiation level'. 

This type of cooling is widely applied in 2D and 3D
numerical simulations of self-gravitating disks \citep{gammie01,
  rice05,baehr15}. It permits numerical experiments to be carried out in a
controlled manner as a function of the cooling time $\beta$. 

We remark that for the linear problem, 
any explicit cooling function $\Lambda=\Lambda(\rho,T)$ can be
mapped to that obtained from the standard function in Eq.
\ref{beta_cool}. We do this in \S\ref{2dppd} where we adopt a more
realistic beta cooling prescription for 2D PPDs.   

For models with beta cooling we assume an equilibrium polytropic relation
\begin{align} 
  P  =
%  \frac{c_{s0}^2\rho_0}{\Gamma}\left(\frac{\rho}{\rho_0}\right)^\Gamma,
  %  \equiv \frac{c_s^2(z)}{\Gamma}\rho,
c_{s0}^2\rho_0\left(\frac{\rho}{\rho_0}\right)^\Gamma,
\end{align}
where $\Gamma$ is the constant polytropic index and $\rho_0 =
\rho(0)$. The vertical
structure is first obtained from Eq. \ref{vert_eq1}---\ref{vert_eq2}
(usually numerically), then inserted into
Eq. \ref{thermal_eq} to infer the required viscosity profile 

\begin{align}\label{alpha_beta_relation}
\alpha(z) = \frac{1}{(\gamma-1)\beta
   q^2}\frac{c_s^2(z)}{c_{s0}^2}\left(1 - \theta\right). 
%  \alpha(z) = \frac{1}{\Gamma(\gamma-1)\beta
%    q^2}\left(\frac{\rho}{\rho_0}\right)^{\Gamma-1} \left(1 - \theta\right). 
\end{align}

\subsubsection{Radiative cooling}\label{rad_cool}
Here we assume energy transfer by radiative diffusion. Then
\begin{align}
  \Lambda &= \nabla\cdot\bm{F},\label{rad_cool1}\\
  \bm{F}   &= -\frac{16\sigma T^3}{3\kappa_d\rho}\nabla T, \label{rad_cool2}
\end{align}
where $\sigma$ is the Stefan-Boltzmann constant and 
$\kappa_d$ is the (dust) opacity. We adopt
\begin{align}\label{opacity_law}
  \kappa_d = \kappa_{d0}T^b,
\end{align}
and take the constant index $b=2$ as appropriate for {\bf blah},
but retain the general notation $b$ to keep track of the opacity.   
%We use radiative cooling when considering 3D PPDs in \S.  

While radiative cooling is arguably more realistic than 
parametrized beta cooling, it complicates our 
problem because it generally implies a vertically non-isothermal
equilibrium disk, and increases the order of the linearized equations. 

In this case we specify a constant viscosity coefficient $\alpha$ and
solve Eq. \ref{vert_eq1}---\ref{thermal_eq}, together with
Eq. \ref{rad_cool1}---\ref{rad_cool2}, as a fourth order system of
ordinary differential equations to obtain $P(z)$, $T(z)$, and hence
$\rho(z)$. 

%\subsection{Steady states}
%\subsubsection{Equilibria with beta cooling}   
%\subsubsection{Equilibria with radiative cooling}
%If $\mathcal{H}_\mathrm{ext}=0$ then viscosity and cooling
%cannot be prescribed independently and we have 
%\begin{align}\label{alpha_beta_relation}
%  \alpha(z) = \frac{1}{\Gamma(\gamma-1)\beta
%    q^2}\left(\frac{\rho}{\rho_0}\right)^{\Gamma-1} \left(1 - \theta\right). 
%\end{align}
%On the other hand, if a time-independent heat source/sink is invoked
%($\mathcal{H}_\mathrm{ext}\neq 0$), then we may choose $\alpha, \beta$
%and $\theta$ separately.   

\subsection{Parameters}
{\bf temporary, maybe move to intro}
For the 3D problem it is convenient to parameterize the strength of
self-gravity via 
\begin{align}
  Q_\mathrm{3D} \equiv \frac{\Omega^2}{4\pi G \rho_0}. 
\end{align}
It is also useful to define the
two-dimensional (2D) Toomre parameter 
\begin{align}
  Q = \frac{c_{s0}\Omega}{\pi G \Sigma}, 
\end{align}
where $\Sigma = \int_{-\infty}^\infty\rho dz$ is the surface density. 


%We are interested in comparing systems with different $\beta$. 
%However, this gives unphysically large values of $\alpha$ for small
%values of $\beta$. In order to compare systems with different $\beta$,
%we adopt the following procedure. We choose parameters  
%$\theta_\mathrm{min}$ and $\alpha_\mathrm{max}$, and evaluate
%Eq. \ref{alpha_beta_relation} at the midplane to define a cooling time
%$\beta_\mathrm{min}$. Then for $\beta\geq\beta_\mathrm{min}$ we use 
%Eq. \ref{alpha_beta_relation} to set $\alpha$ at fixed
%$\theta=\theta_\mathrm{min}$; while for $\beta < \beta_\mathrm{min}$
%we use Eq. \ref{alpha_beta_relation} to set $\theta$ at fixed
%$\alpha(0)=\alpha_\mathrm{max}$.  That is, we raise the reference
%temperature (i.e. irradiation levels) when the imposed cooling is too
%rapid to be sustaind by viscous heating. 


